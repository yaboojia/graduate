\chapter{总结与展望}

\section{主要工作总结}
本文针对资源受限场景下的联邦学习问题,提出了一种基于子模型抽取的解决方案。
具体来说,本文提出了两种不同的子模型抽取方法:基于数据分布感知的子模型抽取方法(FedDSE)
和基于梯度的子模型抽取方法(FedGSE)。
这两种方法分别解决了客户端分布质量不同场景下的资源受限联邦学习问题。

首先,FedDSE方法通过分析客户端数据分布的异质性,提出了一种基于激活值的子模型抽取策略。
该方法在每个训练轮次中,根据模型在客户端本地数据集的激活情况,
动态选择激活值较高的神经元组成子模型。
通过这种方式,FedDSE能够有效减少客户端之间的神经元冲突,提升模型训练的效率和准确性。
实验结果表明,FedDSE在多个基准数据集上均取得了显著的性能提升,
特别是在高数据异质性场景下表现尤为突

其次,FedGSE方法从全局模型与子模型的优化方向出发,
提出了一种基于梯度的子模型抽取策略。
该方法通过在中心侧构建公共数据集,并在全局模型上进行反向传播计算梯度,
选择梯度较大的神经元组成子模型。
通过这种方式,FedGSE能够确保子模型的更新方向与全局模型最为接近,
从而进一步提升模型的训练效果。
实验结果表明,FedGSE在高数据异质性场景下表现尤为优异,显著优于现有的基线方法。

实验验证通过多个基准数据集和广泛的实验设计,
全面评估了FedDSE和FedGSE两种方法的有效性。
在数据集选择上,实验采用了EMNIST、CIFAR10、CIFAR100和TinyImagenet等经典基准数据集,
并在非独立同分布(Non-IID)场景下模拟客户端数据分布的高异质性和低异质性环境,
以还原实际联邦学习场景。
实验结果表明,在低质量客户端分布的情况下,
FedDSE凭借基于激活值的动态神经元选择机制,
有效缓解了客户端之间的数据分布差异对全局模型性能的负面影响,
在全部实验数据集上的准确率相比传统方法有了显著的提高。
在高质量客户端分布场景中,FedGSE通过基于梯度选择的子模型抽取策略,
在实验中表现出显著优势,其子模型的更新方向与全局模型保持高度一致,
使得全局模型的准确率显著提升。
% 在EMNIST数据集上,FedGSE的准确率相比HeteroFL提高了约6\%。
准确率在所有数据集均有显著提升
通过对全局模型与子模型的梯度差异进行分析,
FedGSE有效抑制了梯度误差的累积,
并保证了模型的优化方向一致。
整体实验验证结果充分说明了FedDSE和FedGSE两种方法在解决资源受限场景下
联邦学习问题的实际有效性和优越性能。

\section{未来工作展望}
尽管本文提出的方法在资源受限场景下的联邦学习中取得了显著的进展,
但仍有一些问题和挑战需要进一步研究和解决:

(1)
生成式模型在联邦学习中的应用:本文的FedGSE方法依赖于中心侧的公共数据集来计算梯度。
未来的工作可以探索如何利用生成式模型在边缘设备上生成伪数据,
从而减少对中心侧公共数据集的依赖,进一步提升联邦学习的隐私保护能力。

(2)
模型压缩与加速:尽管本文提出的方法在一定程度上减少了客户端的计算负担,但在实际应用中,
模型的规模和复杂度仍然可能成为瓶颈。
未来的工作可以探索如何在子模型抽取的基础上,
进一步进行模型压缩和加速,以提升联邦学习的整体效率。
% (3)
% 联邦学习的应用场景拓展:本文主要关注的是图像分类任务,
% 但在实际应用中,联邦学习可以应用于更多的场景,
% 例如自然语言处理、推荐系统、医疗诊断等。未来的工作可以探索如何在不同的应用场景下,
% 设计适合的子模型抽取策略,提升联邦学习在各个领域的应用效果。

(3)
在实际场景中,通信成本和计算资源限制是重要的制约因素。
未来可研究如何进一步压缩通信开销或在计算资源更有限的设备上应用本文方法。